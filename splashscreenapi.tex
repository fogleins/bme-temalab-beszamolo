\chapter{Androidos SplashScreen megoldások ismertetése}

% TODO: átfogalmazni
A továbbiakban ismertetett tevékenységek megértéséhez elengedhetetlen
az Android SplashScreen megoldások rövid áttekintése.

\section{API level 30 (Android 11) és korábbi}
Az Android 12-ben bevezetett új API előtt nem volt rendszerszintű támogatás töltőképernyők
fejlesztésére, így a fejlesztők két lehetőség közül választhattak~\cite{splashscreenmigration}:
\begin{itemize}
 \item Egyedi téma bevezetése, amivel a nézet \textit{windowBackground} propertyjét változtatták meg,
 majd állították vissza az alkalmazás betöltését követően az alapértelmezett értékre
 \item Egyedi \textit{Activity} létrehozásával: ezzel egy dedikált osztályt és nézetet hozunk létre
 a töltőképernyő funkció megvalósítására, amely a betöltést vagy timeoutot követően elindítja
 az alkalmazás főképernyőjét % TODO: explicit intenttel, de erre nem biztos, hogy ebben a fejezetben ki kell térni
 % TODO: api a rövidítésjegyzékbe
 % TODO: timeout a szójegyzékbe
 % TODO: Activity a szójegyzékbe: ``An activity is a single, focused thing that the user can do. Almost all activities interact with the user, so the Activity class takes care of creating a window for you in which you can place your UI with''
 %          Forrás: https://developer.android.com/reference/android/app/Activity
\end{itemize}


Az OpenFoodFacts alkalmazásban az utóbbit választották a fejlesztők, ez viszont azt eredményezte,
hogy Android 11 fölötti eszközökön két különböző töltőképernyő jelent meg az alkalmazás indulásakor,
emiatt vált szükségessé az alkalmazás felkészítése az újabb verzióval való helyes működésre.

\section{API level 31 (Android 12) és újabb}
A SplashScreen API bevezetésével maga az operációs rendszer biztosít egységes megoldást
töltőképernyő készítésére. Ez olyan lehetőségekkel bővítette a programozók eszköztárát, ami korábban
nem, vagy csak körülményesen volt megvalósítható (pl. animált ikonok beállítása
a töltőképernyőre).
Az új API arra is lehetőséget ad, hogy a régebbi szoftververziót futtató eszközökön a SplashScreen
compat könyvtár % TODO: app compat hivatkozása
segítségével biztosítsuk a visszafele kompatibilitást.
