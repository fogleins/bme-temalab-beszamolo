\chapter{Androidos SplashScreen megoldások ismertetése}


A továbbiakban ismertetett tevékenységek pontosabb megértéséhez elengedhetetlen
az Androidon elérhető \gls{splashscreen} megoldások rövid áttekintése.

\section{API level 30 (Android 11) és korábbi}
Az Android 12-ben bevezetett új \acrshort{api} előtt nem volt rendszerszintű támogatás töltőképernyők
fejlesztésére, így a fejlesztők két lehetőség közül választhattak~\cite{splashscreenmigration}:
\begin{itemize}
 \item Egyedi \gls{tema} bevezetése: a nézet \textit{windowBackground} tagváltozóját változtatják meg,
 majd állítják vissza az alkalmazás betöltését követően a kívánt értékre
 \item Egyedi \Gls{activity} létrehozása: ezzel egy dedikált osztályt és nézetet hoznak létre
 a töltőképernyő funkció megvalósítására, amely a betöltést vagy időtúllépést követően elindítja
 az alkalmazás főképernyőjét
\end{itemize}

Az \acrlong{off} alkalmazásban az utóbbit választották a fejlesztők, ez viszont azt eredményezte,
hogy Android 11 fölötti eszközökön két különböző töltőképernyő jelent meg az alkalmazás indulásakor,
emiatt vált szükségessé az alkalmazás felkészítése az újabb verzióval való helyes működésre.

\section{API level 31 (Android 12) és újabb}
A \glslink{splashscreen}{SplashScreen} \acrshort{api} bevezetésével maga az operációs rendszer biztosít egységes megoldást
töltőképernyő készítésére. Ez olyan lehetőségekkel bővítette a programozók eszköztárát, ami korábban
nem, vagy csak körülményesen volt megvalósítható (pl. animált ikonok megjelenítése
a töltőképernyőn).
Az új \acrshort{api} arra is lehetőséget ad, hogy a régebbi szoftververziót futtató eszközökön a \gls{splashscreen}
compat könyvtár (mely az \gls{appcompat} könyvtár\footnote{\url{https://developer.android.com/jetpack/androidx/releases/appcompat}} részeként érhető el)
segítségével biztosítsuk a visszafele kompatibilitást.
