\chapter{Összegzés}

A Témalaboratórium tárgy keretein belül alkalmam nyílt betekintést nyerni egy élő,
aktívan fejlesztés alatt álló, széles felhasználói körrel rendelkező projekt fejlesztésébe.
Ezalatt számos új ismeretre tettem szert: ízelítőt kaptam, hogy hogyan lehet kiigazodni egy
összetett szoftver forrásfájljai között, hogyan lehetséges olyan emberekkel közösen dolgozni
egy ilyen kihíváson, akikkel sosem találkoztam, és több ezer kilométerre élnek tőlem.
Megtanultam továbbá, hogy hogyan lehetséges egy alkalmazás forráskódját más hasonló
szoftverben is újrafelhasználni, mi a különböző alkalmazásboltok és ezek sajátosságainak
kezelésének módja. Ezen kívül korábbi tapasztalataimat is bővítettem: jobban elmélyültem
a Kotlin nyelv által nyújtott eszközök, modern nyelvi elemek használatában
(pl. extension function-ök), lehetőségem volt szélesebb körben megismerkedni az Androidos
szoftverfejlesztéssel, valamint áthatóan megismertem néhány könyvtárat
(pl. \gls{splashscreen}, \gls{appcompat}).

A féléves munkám pedig a megszerzett tudáson túl is kifizetődött, ugyanis az általam
készített töltőképernyő-implementáció bekerült az alkalmazás kódbázisába, így több
százezer felhasználó találkozhat vele nap mint nap a program indítása során.
