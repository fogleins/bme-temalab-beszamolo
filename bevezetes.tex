\chapter{Bevezetés, célkitűzések}
Napjainkban egyre gyakrabban szembesülünk az egészséges életmód betartásának nehézségeivel.
A Témalaboratórium tárgy keretein belül olyan szoftverfejlesztési feladattal szerettem volna
foglalkozni, ami kapcsolódik a választott témához (orvosi informatika) és gyakorlati
jelentősége is van, a szoftver használatával megkönnyíthetjük, jobbá tehetjük a felhasználók életét.

Ebben az irányban elindulva, számos különböző projektet átnézve és a konzulensemmel egyeztetve
döntöttem úgy, hogy a félév során az \acrlong{off}\footnote{Weboldal: \url{https://hu.openfoodfacts.org/}}
Android alkalmazását\footnote{GitHub: \url{https://github.com/openfoodfacts/openfoodfacts-androidapp}} fogom fejleszteni.
Miután ez az elhatározás megszületett, igyekeztem olyan feladatot vállalni,
ami megfelel a tárgy követelményeinek, tehát kellő mértékű kihívást jelent
az implementációja, olyan részeket is tartalmaz, melyekkel új ismereteket szerezhetek, és mégis
belefér a tárgy szűkös időkeretébe. Ezeket a szempontokat szem előtt tartva böngésztem a
projekthez tartozó hibajelentéseket és egyéb felhasználói kéréseket, javaslatokat (ún. \textit{feature request}-eket).
Így bukkantam rá egy olyan hibajegyre (\textit{\gls{github} Issue}), amelyben az alkalmazás
töltőképernyőjének (\glslink{splashscreen}{splash screen}) akkori implementációjának módosítását kérték, hogy
az Android 31-es \acrshort{api} szint feletti, azaz Android 12-t vagy annál újabbat futtató eszközökön
is helyesen működjön. \cite{issue}

Mivel korábban az egyetemen már volt lehetőségem betekintést nyerni a mobilos szoftverfejlesztésbe,
és régebbi Android verziót futtató készülékekhez már készítettem hasonló töltőképernyőt, továbbá
sok alkalmazásnál használnak ilyen megoldást, érdekelt, hogy miben újították meg a fejlesztés
folyamatát az új \acrshort{api} bevezetésével.

A félév során tehát szerettem volna kipróbálni magam egy éles projektben, közelebbről megismerkedni
az Androidos szoftverfejlesztéssel, továbbá bővíteni Kotlin nyelvű ismereteimet.
