\chapter{Bevezetés, célkitűzések}
Napjainkban egyre gyakrabban szembesülünk az egészséges életmód betartásának nehézségeivel.
A témalaboratórium tárgy keretein belül olyan szoftverfejlesztési feladattal szerettem volna
foglalkozni, ami kapcsolódik a választott témához (orvosi informatika) és gyakorlati
jelentősége is van, a szoftver használatával megkönnyíthetjük, jobbá tehetjük a felhasználók életét.

Ebben az irányban elindulva, számos különböző projektet átnézve és a konzulensemmel egyeztetve
döntöttem úgy, hogy a félév során az OpenFoodFacts % TODO: hivatkozás az OFF oldalára, github
Android alkalmazását fogom fejleszteni. Miután ez az elhatározás megszületett, igyekeztem olyan
kihívást vállalni, ami megfelel a tárgy követelményeinek, tehát kellő nehézségű az implementációja,
olyan részeket is tartalmaz, melyekkel új ismereteket szerezhetek, és mégis belefér a tárgy szűkös
határidejeibe. Ezeket a szempontokat szem előtt tartva böngésztem a projekthez tartozó
hibajelentéseket és egyéb felhasználói kéréseket, javaslatokat (ún. \textit{feature requests}-eket).
Így bukkantam rá egy olyan hibajegyre (\textit{GitHub Issue}), amelyben az alkalmazás
töltőképernyőjének (Splash screen) akkori implementációjának módosítását kérték, hogy
az Android 31-es API szint feletti, azaz Android 12-t vagy annál újabbat futtató eszközökön
is helyesen működjön.

Mivel korábban az egyetemen már volt lehetőségem betekintést nyerni a mobilos szoftverfejlesztésbe,
és régebbi Android verziót futtató készülékekhez már készítettem hasonló töltőképernyőt, továbbá
sok alkalmazásnál használnak ilyen megoldást, érdekelt, hogy miben újították meg a fejlesztés
folyamatát az új API bevezetésével.
