\chapter{Jelentősebb állomások a~fejlesztés során}
Munkám jelentős részét az Android dokumentációjának tanulmányozása, illetve a különböző lehetőségek
tesztelése tette ki. Fontos volt még megismerni a projekt fejlesztésére vonatkozó irányelveket
(pl. kódformázási konvenciók, pull requestek pontos leírása) \cite{guidelines}.

\section{Ismerkedés a projekttel}
A projekt meghatározását követően igyekeztem megismerni annak felépítését, kideríteni, hogy
pontosan melyik modulokkal kell majd dolgoznom a töltőképernyő frissebb verziójának elkészítéséhez.
Ehhez klónoztam a projekt \gls{github} \glslink{repo}{repositoryját}, és az \gls{androidstudio} fejlesztőkörnyezet
segítségével részletesen tanulmányoztam a projekt struktúráját, a releváns kódrészleteket,
beszereztem a függőségeket, és lefordítottam a programot.

\section{Migrálás az új API-ra, visszafelé kompatibilitás biztosítása}
Android 12-es verziót futtató \gls{emulator} telepítését követően az Android fejlesztői
dokumentációjában fellelhető migrációs útmutatóból tájékozódtam a további feladataimról \cite{splashscreenmigration}.
Az itt leírtakat követve a töltőképernyő helyesen jelent meg az \acrshort{api} 31-et futtató \gls{emulator}on.
Ilyen funkció esetében azonban nem feledkezhetünk meg a visszafelé kompatibilitás támogatásáról
sem, ezért szükséges volt az új könyvtárat nem ismerő eszközökön is tesztelni a programot.
A régebbi \acrshort{api} verziót futtató eszközök esetében azt a megoldást választottam, hogy a már meglévő
töltőképernyő jelenjen meg (azaz egy külön \glslink{splashscreen}{splash screen} \gls{activity} induljon el az alkalmazás
indulásakor), míg az újakon a frissen implementált verzió. A~funkció tesztelése során
megállapítottam, hogy a megoldás alapvetően a megfelelő eredményt adja, azonban -- ahogy az a
\gls{github} issue-ban is szerepelt \cite{issue} -- az éjszakai módban nem az elvárt módon működik: sötét helyett
fehér háttérszínnel jelenik meg a képernyő. A hibajegyhez mellékelt képernyőképen szereplőhöz
hasonló kinézetű töltőképernyőt az \gls{emulator}on nem sikerült reprodukálni, ott az éjszakai mód
bekapcsolását követően is fehér háttérszínnel indult el a töltőképernyő, csak az alkalmazás
további részei használták az éjszakai témát. Mivel Android 12-t futtató eszköz nem állt
rendelkezésemre, ezért valódi hardveren nem tudtam tesztelni az programot.

\begin{lstlisting}[frame=single, language=Kotlin, caption=A \gls{splashscreen} compat könyvtár
    segítségével visszafele kompatibilitást támogató kódrészlet]
val splashScreen: SplashScreen? =
    if (Build.VERSION.SDK_INT >= Build.VERSION_CODES.S) {
        installSplashScreen()
    } else {
        null
    }
splashScreen?.setKeepOnScreenCondition { true }
\end{lstlisting}


A munkám során alkalmam nyílt megismerni egy, az Androidos szoftverfejlesztésben használt
megoldást az alkalmazás különböző verzióinak egyszerű kezelésére és előállítására. Ebben a projektben ez
különösképpen nagy jelentőséggel bír, hiszen nem csak arról van szó, hogy az alkalmazást több
alkalmazásboltban is terjesztik (pl. Google Play, F-Droid), hanem ugyanarra a kódbázisra
három különböző alkalmazást (\acrfull{off}, \acrfull{obf}, \acrfull{opff})
is építenek. Ez azért is lehetséges, mert az alkalmazás lényegében csak egy felhasználói felület
az \acrlong{off} (és a többi verzió) adatbázisának elérésére. Így maga a kliens oldali alkalmazás
az ikonok és egyéb felhasználói elemek kivételével megegyezik a három verzió között.
A build variants használata lehetővé teszi, hogy mindössze egyetlen Android appot
készítsenek, amit különféle verziókra (flavor) lehet lefordítani. Eleinte nehézséget okozott,
hogy alapértelmezetten a \Gls{git} \gls{repo} klónozását követően az \acrlong{obf} alkalmazás
települt. Mivel az alkalmazás szinte teljesen megegyezik az \acrlong{off} alkalmazással, a
tesztelés során ez nem jelentett gondot, de a \glslink{splashscreen}{splash screenen} megjelenő ikon más, ezért az éjszakai
mód teszteléséhez már az \acrlong{off}-et volt szükséges telepíteni. Némi kutatómunka után
találtam rá arra a lehetőségre \gls{androidstudio}ban, amivel a fordítandó build variant-ot lehet
kiválasztani. Így a továbbiakban a Google Play-re targetelt \acrlong{off} alkalmazással
folytattam a fejlesztést.

\section{Implementáció továbbfejlesztése: sötét mód támogatása}
Az alapműködés implementációját követően a munkám az éjszakai módban látható töltőképernyő helyes megjelenítésére irányult.
Sikerült kölcsönkérnem egy Android 12-es verziót futtató eszközt, azonban ezen sem sikerült
előállítanom a \gls{github} issue-ban \cite{issue} szereplő módon a hibát, így folytattam az okának felkutatását
a kódban és az interneten egyaránt. Rövid utánajárást követően kiderült, hogy a
\textit{styles.xml} fájlból nem volt megadva \textit{night} \gls{resmod}vel
(resource qualifier) ellátott verzió. Ezt a hiányosságot pótoltam, létrehoztam egy éjszakai
módban helyesen megjelenő stílus erőforrásfájlt. Ebben a háttérszín megváltoztatásán túl szükséges
volt a SplashScreen \acrshort{api}-által biztosított témából egy sajátot leszármaztatni, mely rendelkezik
az ikon helyes megjelenítéséhez szükséges propertykkel. Az így született \gls{tema} az alábbi:

\begin{lstlisting}[frame=single,language=xml,emph={style,item},
    emphstyle=\color{BurntOrange}\textbf,stringstyle=\color{OliveGreen}\textbf]
<style name="SplashTheme"
  parent="Theme.SplashScreen.IconBackground">
    <item name="windowSplashScreenBackground">
        @color/grey_800
    </item>
    <item name="windowSplashScreenIconBackgroundColor">
        @color/grey_400
    </item>
    <item name="postSplashScreenTheme">
        @style/Theme.AppCompat.DayNight
    </item>
    <item name="windowSplashScreenAnimatedIcon">
        @mipmap/ic_launcher_round
    </item>
</style>
\end{lstlisting}

Ez azonban nem jelentett megoldást, mert a \gls{tema}, amiből az \acrshort{api} 31 feletti helyes működésért
le kell származtatnunk a saját \glslink{tema}{témánkat}, nem támogatja az \gls{appcompat}\gls{activity} használatát, ami
az \acrlong{off} \glslink{activity}{activity}\-/implementációjának ősosztálya. Így a \gls{splashscreen} \acrshort{api} által biztosított
\gls{tema} kizárólagos használatával az alkalmazás a töltőképernyő megjelenítését követően hibára futott,
ez \az{\ref{fig:appcompatcrash}} ábrán látható.
\kep[0.5]{include/theme-appcompat-crash.png}{\textit{AppCompat} \gls{tema} használata nélkül
az alkalmazás összeomlik}{appcompatcrash}

% TODO: StackOverflow és Android Developer hivatkozása?
A probléma megoldására az Android fejlesztői oldalán elérhető leírás
nyújtott iránymutatást \cite{splashscreenmigration}. A \textit{postSplashScreenTheme} property megfelelő beállítását követően
az alkalmazás már megfelelően működött, a töltőképernyő után a főképernyő helyesen jelent meg.
Így az éjszakai módban használt \gls{tema} kódja az alábbi:

\begin{lstlisting}[frame=single,language=xml,emph={style,item},
    emphstyle=\color{BurntOrange}\textbf,stringstyle=\color{OliveGreen}\textbf]
 <style name="SplashTheme"
   parent="Theme.SplashScreen.IconBackground">
    <item name="windowSplashScreenBackground">
        @color/grey_800
    </item>
    <item name="windowSplashScreenIconBackgroundColor">
        @color/grey_400
    </item>
    <item name="postSplashScreenTheme">
        @style/Theme.AppCompat.DayNight
    </item>
</style>
\end{lstlisting}

A fent definiált \gls{tema} az alábbi módon jelenik meg a gyakorlatban (\ref{fig:darktheme} ábra):
\kep[0.11]{include/darktheme.png}{A töltőképernyő éjszakai módban, Android 12-n}{darktheme}


\section{Egyezetés a projekt fejlesztőivel}
A funkciók tesztelését követően kommunikációt folytattam az alkalmazás fejlesztőivel a pull request lehetséges
javításairól, illetve a módosításaim kódbázisba kerüléséről \cite{pullrequest}. Az alkalmazás egyik fejlesztője egy import
eltávolítását javasolta, rövid utánajárást követően \cite{extensionfunction} én viszont arra jutottam, hogy ez nem
hagyható el, ugyanis egy ún. Kotlin \textit{extension function}-ről van szó, amire az import
nélkül nem tudunk hivatkozni. A Kotlinban támogatott extension function-ök olyankor lehetnek
hasznosak, amikor egy már -- nem általunk -- megírt osztály lehetőségeit szeretnénk bővíteni.
Ilyen megoldást alkalmaztak a \gls{splashscreen} könyvtár \cite{splashscreenreference} fejlesztői is, az Android által biztosított
\gls{activity} osztályt bővítették ki egy \textit{installSplashScreen} metódussal:

\begin{lstlisting}[frame=single,language=Kotlin,emph={style,item}
    ,caption=Részlet a SplashScreen könyvtár kódjából az extension function-nel]
 public fun Activity.installSplashScreen():
            SplashScreen {
    // ...
 }
\end{lstlisting}
\kep[0.56]{include/gh-discussion-light.png}{Egyeztetés a projekt egyik fejlesztőjével az extension function importjáról}{ghdiscussion}

Végül a projekt másik fejlesztője is az enyémmel azonos következtetésre jutott, az import
nem hagyható el, így a módosításaimat elfogadták, és bekerültek az alkalmazás kódbázisába.

% TODO: backwards compatibility támogatása
%       sötét téma
%       https://developer.android.com/develop/ui/views/launch/splash-screen/migrate#best-practices
%       https://developer.android.com/reference/android/window/SplashScreen
%       https://developer.android.com/develop/ui/views/launch/splash-screen
%       https://github.com/openfoodfacts/openfoodfacts-androidapp/pull/4871
%       https://github.com/openfoodfacts/openfoodfacts-androidapp/issues/4548
%       https://kotlinlang.org/docs/extensions.html#scope-of-extensions
% repository fogalomjegyzékbe
