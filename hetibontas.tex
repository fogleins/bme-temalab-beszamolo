\chapter{Heti munkák}
Munkám jelentős részét az Android dokumentációjának tanulmányozása, illetve a különböző lehetőségek
tesztelése tette ki. Fontos volt még megismerni a projekt fejlesztésére vonatkozó irányelveket
(pl. kódformázási konvenciók, pull requestek pontos leírása).

\section{Első hét}
A projekt meghatározását követően igyekeztem megismerni annak felépítését, kideríteni, hogy
pontosan melyik modulokkal kell majd dolgoznom a töltőképernyő frissebb verziójának elkészítéséhez.
Ehhez klónoztam a projekt GitHub repositoryját, és az Android Studio fejlesztőkörnyezet
segítségével részletesen tanulmányoztam a projekt struktúráját, a releváns kódrészleteket,
beszereztem a függőségeket, és lefordítottam a programot.

\section{Második hét}
Android 12-es verziót futtató emulátor telepítését követően az Android fejlesztői
dokumentációjában fellelhető migrációs útmutatóból tájékozódtam a további feladataimról.
Az itt leírtakat követve a töltőképernyő helyesen jelent meg az API 31-et futtató emulátoron.
Ilyen funkció esetében azonban nem feledkezhetünk meg a visszafelé kompatibilitás támogatásáról
sem, ezért szükséges volt az új API-t nem ismerő eszközökön is tesztelni a megoldást.
A régebbi API verziót futtató eszközök esetében azt a megoldást választottam, hogy a már meglévő
töltőképernyő jelenjen meg (azaz egy külön splash screen activity induljon el az alkalmazás
indulásakor), míg az újakon a frissen implementált verzió. A~funkció tesztelése során
megállapítottam, hogy a megoldás alapvetően az elvárt eredményt adja, azonban -- ahogy az a
GitHub issue-ban is szerepelt -- az éjszakai módban nem az elvárt módon működik: sötét helyett
fehér háttérszínnel jelenik meg a képernyő. A hibajegyhez mellékelt képernyőképen szereplőhöz
hasonló kinézetű töltőképernyőt az emulátoron nem sikerült reprodukálni, ott az éjszakai mód
bekapcsolását követően is fehér háttérszínnel indult el a töltőképernyő, csak az alkalmazás
további részei használták az éjszakai témát. Mivel Android 12-t futtató eszköz nem állt
rendelkezésemre, ezért valódi hardveren nem tudtam tesztelni a megoldást.

% TODO: megoldás szóismétlés
A heti munkám során alkalmam nyílt megismerni egy, az Androidos szoftverfejlesztésben elterjedt
megoldást az alkalmazás különböző verzióinak egyszerű előállítására. Ebben a projektben ez
különösképpen nagy jelentőséggel bír, mert nem csak arról van szó, hogy az alkalmast több
alkalmazásboltban is terjesztik (pl. Google Play, F-Droid), hanem ugyanarra a kódbázisra
három különböző alkalmazást (OpenFoodFacts (OFF), OpenBeautyFacts (OBF), OpenPetFoodFacts (OPF)) % TODO: ez OPF vagy OPFF?
is építenek. Ez azért is lehetséges, mert az alkalmazás lényegében csak egy felhasználói felület
az OpenFoodFacts (és a többi verzió) adatbázisának elérésére. Így maga a kliens oldali alkalmazás
az ikonok és egyéb felhasználói elemek kivételével megegyezik a három verzió között.
A build variants % TODO ...

% Build variants: az OpenFoodFacts csapata az élelmiszerek mellett többféle termékkategóriához
% készít szoftvert (pl. szépészeti készítmények, állateledelek). Mivel mindezek a verziók szoftveresen
% nagyon hasonlóak, csak más-más backendet használnak, ezért mindössze egyetlen Android appot
% készítenek, melyet lehetőség van különféle verziókra (flavor) fordítani. Eleinte nehézséget okozott,
% hogy alapértelmezetten a git repository klónozását követően az OpenBeautyFacts alkalmazás
% települt. Mivel az alkalmazás szinte teljesen megegyezik az OpenFoodFacts alkalmazással, a
% tesztelés során ez nem jelentett gondot, de a splash screenen megjelenő ikon más, ezért az éjszakai
% mód teszteléséhez már az OpenFoodFacts-et volt szükséges telepíteni. Némi kutatómunka után
% találtam rá arra a lehetőségre Android Studioban, amivel a fordítandó build variant-ot lehet
% kiválasztani. Így a továbbiakban a Google Play-re targetelt OpenFoodFacts alkalmazással
% folytattam a fejlesztést.

% TODO: ez még szerkesztésre szorul, csak át lett emelve a második hét alfejezetből,
%       de ide való
\section{Harmadik hét}
Ezen a héten a munkám az éjszakai módban látható töltőképernyő helyes megjelenítésére irányult.
Sikerült kölcsönkérnem egy Android 12-es verziót futtató eszközt, azonban ezen sem sikerült
előállítanom a GitHub issue-ban szereplő módon a hibát, így folytattam a hiba okának felkutatását
a kódban és az interneten egyaránt. Rövid utánajárást követően kiderült, hogy a
\textit{styles.xml} fájlból nem volt megadva \textit{night} erőforrás-módosítóval
(resource modifier) ellátott verzió. Ezt a hiányosságot pótoltam, létrehoztam egy éjszakai
módban helyesen megjelenő stílus erőforrást. Ebben a háttérszín megváltoztatásán túl szükséges
volt a SplashScreen API-által biztosított témából egy sajátot leszármaztatni, mely rendelkezik
az ikon helyes megjelenítéséhez szükséges propertykkel. Az így született téma az alábbi:

\begin{lstlisting}[frame=single,language=xml]
<style name="SplashTheme"
  parent="Theme.SplashScreen.IconBackground">
    <item name="windowSplashScreenBackground">
        @color/grey_800
    </item>
    <item name="windowSplashScreenIconBackgroundColor">
        @color/grey_400
    </item>
    <item name="postSplashScreenTheme">
        @style/Theme.AppCompat.DayNight
    </item>
    <item name="windowSplashScreenAnimatedIcon">
        @mipmap/ic_launcher_round
    </item>
</style>
\end{lstlisting}

Ez azonban nem jelentett megoldást, mert a téma, amiből az API 30 feletti helyes működésért
le kell származtatnunk a saját témánkat, nem támogtja az AppCompatActivity használatát, ami
az OpenFoodFacts activity\-/implementációjának ősosztálya. Így a SplashScreen API által biztosított
téma kizárólagos használatával az alkalmazás a töltőképernyő megjelenítését követően crashelt,
ez \az{\ref{fig:appcompatcrash}} ábrán látható.
\kep[0.5]{include/theme-appcompat-crash.png}{\textit{AppCompat} téma használata nélkül
az alkalmazás összeomlik}{appcompatcrash}

A probléma megoldására a StackOverflow-n és az Android fejlesztői oldalán elérhető leírás
nyújtott iránymutatást. A postSplashScreenTheme property megfelelő beállítását követően
az alkalmazás már megfelelően működött, a töltőképernyő után a főképernyő helyesen jelent meg.
Így az éjszakai módban használt téma kódja az alábbi:

\begin{lstlisting}[frame=single,language=xml]
 <style name="SplashTheme"
   parent="Theme.SplashScreen.IconBackground">
    <item name="windowSplashScreenBackground">
        @color/grey_800
    </item>
    <item name="windowSplashScreenIconBackgroundColor">
        @color/grey_400
    </item>
    <item name="postSplashScreenTheme">
        @style/Theme.AppCompat.DayNight
    </item>
</style>
\end{lstlisting}

A fent definiált téma az alábbi módon jelenik meg a gyakorlatban (\ref{fig:darktheme} ábra):
\kep[0.11]{include/darktheme.png}{A töltőképernyő éjszakai módban, Android 12-n}{darktheme}


\section{Negyedik hét}
A negyedik héten kommunikációt folytattam az alkalmazás fejlesztőivel a pull request lehetséges
javításairól, illetve ennek merge-öléséről. A módosításaim végül elfogadásra kerültek, és
bekerültek az alkalmazás kódbázisába.
% TODO: backwards compatibility támogatása
%       sötét téma
%       https://developer.android.com/develop/ui/views/launch/splash-screen/migrate#best-practices
%       https://developer.android.com/reference/android/window/SplashScreen
%       https://developer.android.com/develop/ui/views/launch/splash-screen
% GitHub fogalomjegyzékbe
% repository fogalomjegyzékbe
% Android Studio fogalomjegyzékbe
% emulátor fogalomjegyzékbe
% TODO: \begin{verbatim} helyett \begin{xmlcode}
% TODO: AppCompatActivity hivatkozása
